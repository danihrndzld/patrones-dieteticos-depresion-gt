\PassOptionsToPackage{hidelinks}{hyperref}
\documentclass[jou]{apa7}

% Packages for basic functionality and formatting
\usepackage[T1]{fontenc}
\usepackage[utf8]{inputenc}
\usepackage[english]{babel}

% Package for graphics and custom commands for images
\usepackage{graphicx}
\usepackage{floatrow} % Enhanced control over figure and table placements

% Custom command for including graphics with a max size
\newcommand{\includegraphicsmax}[2][]{%
	\includegraphics[width=\textwidth,height=0.25\textheight,keepaspectratio,#1]{#2}%
}

% Define parameters for subfigure size and vertical space
\newcommand{\subfigwidth}{0.2\textwidth}
\newcommand{\subfigvspace}{0.1em}

% Subfigure and caption packages
\usepackage{subcaption}

% Table formatting packages
\usepackage{tabularx}
\usepackage{booktabs}
\usepackage{array}

% Text and typography enhancements
\usepackage{textgreek} % For using Greek characters in text
\usepackage{microtype} % Improves spacing and reduces overfull hbox warnings

% Hyperlinks and citations management
\usepackage{hyperref} % Always load after other packages

% Bibliography management with APA style
\usepackage[style=apa,sortcites=true,sorting=nyt,backend=biber]{biblatex}
\DeclareLanguageMapping{english}{english-apa}
\addbibresource{camila.tesis.ufm.bib}

% Quotation package
\usepackage{csquotes}

% Title, authors, and abstract
\title{Dietary Patterns and Depressive Symptoms in Young Guatemalan Women: An Analysis of Specific Correlations}
\shorttitle{Diet and Depressive Symptoms in Guatemalan Women}

\author{
	\addORCIDlink{Camila Heredia, M.D.}{0009-0008-9550-9083} and \addORCIDlink{Lic. María Andrée Neumann}{0009-0001-2531-6058}
}

\affiliation{Graduate School, Universidad Francisco Marroquín}

\leftheader{Heredia, Neumann}

\abstract{This study examines the relationship between diet and depression in young Guatemalan women. Using a quantitative, non-experimental, cross-sectional correlational design, a sample of $30$ participants was analyzed through the Beck Depression Inventory and a dietary questionnaire. The results revealed significant correlations between dietary patterns and depressive symptoms. A strong negative association was found between fruit consumption and symptoms such as loss of pleasure $(r = -0.49, p = 0.006)$ and suicidal thoughts $(r = -0.48, p = 0.007)$. Processed food consumption was positively correlated with symptoms such as pessimism and loss of interest. The findings suggest a potential protective effect of a fruit-rich diet against depression. The study highlights the importance of considering dietary interventions in the prevention and treatment of depression in young women.}

\keywords{Keywords: Beck, mood, nutrition}

\authornote{
	This work is presented as part of the Research Methodology I course taught by Professor Regina Fernández Morales
	during the fourth term of 2023 at UFM. The project presents no conflicts of interest, and the content is original in terms of the literature review, objectives, and methodology proposed.

	Communication with the authors should be made through any of the following emails:
	\href{mailto:camilah@ufm.edu}{\nolinkurl{camilah@ufm.edu}} or
	\href{mailto:mneumann@ufm.edu}{\nolinkurl{mneumann@ufm.edu}}
}

\usepackage{times} % Times font, often preferred for academic papers


\begin{document}
	\maketitle

	%	\tableofcontents


	\section{Contextual Framework}\label{marco-contextual}

	The research was conducted on young women in Guatemala City. The decision to collect samples at the Salud UFM clinics and process them at LABOCLIP was based on ease of access and the willingness of these institutions to participate in the study. The Salud UFM clinic, being an institution that collaborates in the collection of samples for its students' research, and LABOCLIP, a laboratory known for its willingness to work with the academic community, provided a conducive environment for carrying out this research.\\

	The study conducted is of significant current relevance given the increasing prevalence of depression globally. Depression is a common illness worldwide, estimated to affect more than 300 million people. On April 7, 2017, in commemoration of World Health Day, the WHO launched an annual campaign titled: "Let's Talk About Depression," highlighting the significant concern this topic generates \parencite{Toshi2022}.\\

	Moreover, the results of this research could significantly contribute to the prevention and treatment of comorbidities related to diet and mental health. If a negative or non-significant association is found, the findings would still be valuable in guiding future research in this area.

	Although similar studies have been conducted in various locations, there is a lack of similar research in the Guatemalan population in particular. In general, Guatemala has been a country where research on depression has been less prevalent compared to other regions, which underscores the importance of this study.\\

	Depression is a highly comorbid condition, meaning it often occurs alongside other diseases. Understanding the relationship between diet and depression could help improve the management of these comorbid medical conditions.


	\subsection{Conceptual Framework}
	Diet can be defined as the regular consumption of food and drink, which can have significant impacts on physical and mental health. Certain dietary patterns are characterized by increased intake of fruits, vegetables, and other nutrient-rich foods that can contribute to overall well-being.\\

	The role of diet in mental health has been increasingly recognized. Specific nutrients, such as antioxidants, vitamins, and minerals found in fruits and vegetables, are crucial for maintaining optimal neurological function and can potentially protect against mental health issues.\\

	Furthermore, dietary habits can influence long-term health outcomes. Diets high in processed foods, sugars, and unhealthy fats have been linked to various negative health effects, including the exacerbation of mental health disorders such as depression.\\

	Nutrition plays a fundamental role in maintaining mental health. Some nutrients, such as antioxidants found in fruits and vegetables, have properties that can support mental well-being. A balanced diet rich in these nutrients can help prevent related mental health issues.

	Similarly, stress is a factor that can influence mental health. Chronic stress can negatively impact dietary habits, leading to poor nutrition, which in turn can exacerbate mental health issues. Proper stress management can be an effective tool to promote better dietary habits and mental health.

	Among the negative impacts of poor diet is an increased risk of mental health disorders, which highlights the importance of addressing dietary habits as part of a holistic approach to health.\\

	There are nutritional factors that can affect or modulate mental health. These include the total intake of calories and the balance of macronutrients and micronutrients. Research suggests that a well-balanced diet can be useful in maintaining mental health and preventing disorders such as depression.\\

	In recent years, the connection between diet and mental health has been increasingly studied, showing that all aspects of health are interrelated. Therefore, the importance of studying the human being holistically, considering both physiological and mental aspects, has been emphasized.

	In addition, systematic reviews have suggested a strong connection between diet and depression. Poor dietary habits could play a role in the development of depression, and depression could further worsen dietary choices. The word "depression" is often used to describe many different things. It can define a fleeting mood, perhaps an external physical appearance of sadness, or for others, a diagnosable clinical disorder. Each year, millions of adults suffer from clinically diagnosed depression, a mood disorder that often affects personal, vocational, social, and health functioning \parencite{APA2014}.



\subsection{State of the Art}

The study of the relationship between diet and depression has been of growing interest over the past decade, even though it remains a developing area. The connection between mental and physical health has been a constant concern in medicine and psychology, and more recently, the role of diet has emerged as a potential mediator in this relationship.

The role of diet in mental health has captured the attention of the scientific community due to its link with various chronic diseases and mental health disorders. Although poor dietary habits can originate from multiple causes, such as lifestyle factors, recent emphasis has been placed on investigating the impact of diet as a potential contributor to mental health issues. This concern arises because poor nutrition can increase the risk of conditions such as insulin resistance, diabetes, metabolic syndrome, and cardiovascular diseases, which are often associated with mental health disorders. Therefore, understanding and modulating diet may be key to preventing and managing such diseases.

Some nutrients may have beneficial properties for mental health, while others may be detrimental. Therefore, it is important to know which nutrients can support mental health and in what quantity they should be consumed.

It is essential to mention that most studies employed multiple linear regression for their analyses and adjusted for factors such as gender, age, socioeconomic status, body mass index (BMI), among others. These findings reinforce the relevance of diet concerning mental health outcomes in young populations.

According to \parencite{Farre2020}, nutrition is essential not only for our survival and growth but also for maintaining the homeostasis of different components of the mucosal barrier. Research has shown that nutrients play crucial roles in: (1) maintaining the intestinal epithelium, promoting cell growth, homeostasis, and functions; (2) regulating the function of the intestinal epithelial barrier; (3) modulating intestinal immunity; and surprisingly, (4) nutritional supplementation could improve mucosal abnormalities present in patients with gastrointestinal disorders.\\

This overview reinforces the relevance of nutrients in the homeostasis of the mucosal barrier and the maintenance of normal intestinal physiology. More well-designed clinical trials are needed to confirm the possibility of nutritional supplementation as a treatment for patients with mucosal barrier dysfunction, including those with diseases such as celiac disease, non-celiac gluten sensitivity, irritable bowel syndrome, and functional dyspepsia \parencite{Farre2020}.

%%% REMOVED CITATION: \parencite{ridkerHighsensitivityCreactiveProtein2004}

The following are studies that have explored this connection and their conclusions. In a systematic review, the interaction between diet and depression was explored in 109 studies, most of which were rated moderate to high in quality \parencite{Toenders2022}.\\

When examining dimensional measures of depressive symptoms, most studies found no significant correlation between the severity of depression and key dietary patterns in healthy young people, suggesting that clinical levels of depression may be necessary to observe a significant impact from dietary changes \parencite{Toenders2022}.\\

From a longitudinal perspective, variability has been observed in responses following different therapeutic interventions. For example, a study on cognitive-behavioral therapy (CBT) for insomnia in patients with knee osteoarthritis examined its effects on mental health outcomes \parencite{munPreliminaryExaminationEffects2024}. In the case of breast cancer patients, research evaluated the impact of brief stress management interventions, including CBT, on mental health outcomes \parencite{taubEffectsRandomizedTrial2019}. Additionally, in patients with heart failure, the effects of CBT for insomnia on autonomic function and mental health outcomes were analyzed \parencite{redekerEffectsCognitiveBehavioral2020}. These studies suggest that the effects of psychological interventions on mental health are not uniform and may vary depending on the condition treated and the specific mental health outcomes measured. The observed divergences could be attributed to differences in treatment protocols, studied populations, or the complex interactions between the various components of the mental health and overall well-being.


\subsection{Problem Statement}

The relationship between diet and depression, as previously mentioned, has been the subject of multiple investigations globally, with robust evidence linking dietary patterns and depressive episodes in various populations. However, most of these studies have focused on specific populations, leaving many regions, such as Central America, with a notable lack of data in this area.\\

Guatemala, despite having a population with unique sociodemographic, genetic, and environmental traits, lacks research addressing the interaction between diet and depression. This knowledge gap is exacerbated when considering how Guatemalan socioeconomic and demographic factors modulate this relationship. In the Guatemalan context, where genetic diversity and variability in environmental and socioeconomic factors are significant, it is imperative to fill this informational gap to offer more contextualized interventions and treatments.

This research seeks to answer the pressing question: How are dietary habits related to the severity of depressive symptoms in the population of young Guatemalan women?


\section{Objectives}\label{objetivos}

\subsection{General Objective}\label{objetivo-general}

To analyze the relationship between the severity of depressive symptoms and dietary habits in the population of young Guatemalan women.

\subsection{Specific Objectives}\label{objetivos-especuxedficos}

\begin{enumerate}
	%%% REMOVED OBJECTIVE: \item Identify and quantify the main inflammatory markers present in young Guatemalan women.
	\item Identify depressive symptoms using the Beck Depression Inventory.
	\item Determine or characterize the dietary habits of young Guatemalan women through a questionnaire.
	\item Establish the degree of correlation between specific dietary habits and the severity of depressive symptoms.
	\item Describe the distribution of depressive symptoms in the population of young Guatemalan women based on their dietary patterns.
\end{enumerate}



\section{Materials and Methods}\label{materiales-y-muxe9todos}

\subsection{Research Design}\label{diseuxf1o-investigaciuxf3n}

The design of this research was quantitative, non-experimental, and cross-sectional correlational. This approach was carefully chosen to objectively study the correlation between diet and mood in the sample.\\

\subsubsection{Approach}
In this study, the quantitative approach was employed to measure and analyze the variables in a numerical and statistical manner. The primary interest was in measuring how variations in dietary habits were described by the independent variable(s), i.e., how diet and mood could influence each other. This quantitative approach allowed for an objective and precise evaluation (particularly in the $R^2$ coefficient) of the relationships between these variables, facilitating data-based conclusions.\\

\subsubsection{Scope}
The scope of this research was correlational.
The aim was to understand how diet was related to mood. By analyzing these correlations, it was hoped to discover whether statistically significant associations existed that could suggest (though not prove) a causal relationship or influence between the factors.

\subsubsection{Techniques}
To obtain reliable and accurate data, a combination of techniques was employed. The Beck Depression Inventory, a validated and reliable instrument for measuring mood, was used to assess the participants. Additionally, a dietary questionnaire was administered to gather detailed information about the participants' dietary habits.

\subsubsection{Dietary Questionnaire}
Complementarily, to gather detailed information about the participants' diets, a questionnaire specifically designed for this study was applied. This combined techniques approach allowed for a holistic and systematic analysis of the relationships between mood and diet.


\subsection{Instruments}\label{instrumentos}

A combination of meticulously selected instruments was used to accurately assess the interrelationships between diet and mood in young women in Guatemala City.\\

\subsubsection{Dietary Habits Questionnaire}
This questionnaire was designed to obtain detailed and relevant information about the participants' food consumption and dietary patterns. This questionnaire, following best practices in questionnaire design, included carefully formulated questions to ensure the accuracy and relevance of the selected data.\\

\subsubsection{Beck Depression Inventory}
This validated and widely recognized and used instrument was applied to measure the severity of the participants' depressive symptoms. This scale facilitated a quantitative and comparative evaluation of mood states, allowing for the correlation of these data with dietary habits. This questionnaire consists of 21 groups of statements.\\

The total score was obtained by summing each of the items, with 0 being the minimum score and 63 the maximum. The scoring norms in the Mexican population are as follows: from 0 to 5 points, minimal anxiety; from 6 to 15, mild anxiety; from 16 to 30 points, moderate anxiety, and from 31 to 63, severe anxiety \parencite{beckDepressionCausesTreatment2009}.

%%% REMOVED SECTION: \subsubsection{Laboratory Tests for Inflammatory Markers}
%%% REMOVED CONTENT: As part of the sampling process, clinical tests were performed to identify and quantify the inflammatory markers mentioned earlier: C-reactive protein and erythrocyte sedimentation rate. These tests generated an objective base of clinical data that was used in conjunction with the other instruments to allow for a comprehensive analysis of the interactions between the different variables.

\subsection{Sample and Population}\label{muestra-y-poblaciuxf3n}

The population of this research consisted of a homogeneous sample of 30 young women residing in Guatemala City. Thirty women were chosen because it was assumed that this sample would achieve normality \parencite{Hernandez2018}. The participants were selected to provide a representative perspective on the interactions between diet and mood.


The study considered the following inclusion criteria:
\begin{enumerate}
	\item Women
	\item Aged 25 to 30 years
	\item Residing in Guatemala City
	\item Willingness to participate in the study
\end{enumerate}

Additionally, the following exclusion criteria were identified: a) Women with chronic illnesses b) Women with autoimmune diseases c) Pregnant women, to avoid any influence of these conditions on the study results. Clarifying these exclusions, participants were warned that if they did not meet the exclusions, the researchers would not be held responsible, as they signed and read the informed consent. This sample and population selection methodology ensured that the research was accurate, relevant, and replicable.

\subsection{Selection and Definition of Variables}\label{selecciuxf3n-y-definiciuxf3n-de-variables}

In this study, variables encompassing psychological and nutritional aspects were selected. Thus integrating both quantitative and qualitative measures to achieve a comprehensive and thorough analysis.\\

\subsubsection{Beck Depression Inventory}
This is an ordinal measure that evaluated the severity of depressive symptoms. The scale is a validated and recommended psychometric tool, and its inclusion allowed for the quantification of the participants' mood in a standardized and reliable manner.

\subsubsection{Dietary Questionnaire}
This is a categorical variable designed to assess the dietary habits of the participants as well as their quality. This questionnaire helped capture detailed information about food consumption and dietary patterns, which was crucial for investigating the relationship between diet and mood.

\section{Hypothesis}\label{hipuxf3tesis}

There is a significant bidirectional relationship between diet and depression, which influences mood and overall well-being.

Based on a comprehensive review of the literature and understanding of the interactions between diet and mood, the following hypothesis is proposed: "There is a statistically significant bidirectional relationship between diet and depression, which influences overall well-being." This hypothesis is based on the premise that dietary patterns do indeed contribute to mood disturbances and, in turn, influence overall mental and physical health, being bidirectional, the reverse is also true. The confirmation or refutation of this hypothesis will provide valuable information on how these factors interact and influence each other.

\section{Data Collection Procedure}\label{procedimiento-para-recolecciuxf3n-de-datos}

After the topic and methodology for this research were approved, it was necessary to conduct a detailed literature review to validate the importance of the research. It was ensured that the background information was up-to-date and highly reliable.

It was ensured that the matrix instrument was reliable for selecting the searched articles. For this, it was necessary to specify the different inclusion and exclusion criteria. The criteria for article selection were of utmost importance to ensure specificity and clarity. Additionally, research involving young Guatemalan women investigating diet and mood, and finding specific instruments such as scales to measure mood were necessary.

The results of the articles deemed pertinent and necessary for the research were used. Two investigators responsible for this research reviewed the data to minimize bias and extract the most accurate data.

The research data were grouped based on diet and mood scale. A scale was used to measure mood and an interview to analyze the diet of each research participant, using them to understand the relationship between dietary habits and mood.


\subsection{Statistical Analysis and Data Processing}

Statistical analysis and data processing formed the backbone of this research; because by providing the necessary quantitative foundation, it was possible to evaluate the hypothesis and conclude. Below is a detailed account of the methodological approach and the various statistical techniques used to explore and understand the interrelationships between the study variables. Through a rigorous and systematic process, it was proposed to discover statistically significant patterns, trends, and correlations that shed light on the complex dynamics between diet and mood.


\subsection{Description and Justification of Methods and Analysis Techniques}\label{descripciuxf3n-y-justificaciuxf3n-de-muxe9todos-y-tuxe9cnicas-de-anuxe1lisis}

The selection of statistical methods and analysis techniques for this study was guided by statistical principles and based on a deep understanding of the nature of the variables involved. Since the goal was to examine possible correlations between variables: dietary habits and mood states, correlation and regression techniques were employed to investigate the relationships between the variables. Correlation allowed us to determine the strength and direction of the relationship between two variables, while regression analysis helped to understand how the variation in the dependent variable was described by the independent variables.\\

Additionally, considering the nature of the data collected, which included ordinal measures (Beck scale) and categorical measures (dietary patterns), methods that fit these characteristics were used. Among them were normality tests to evaluate the distribution of the sample; a crucial step in selecting the most appropriate statistical tests. Furthermore, analysis of variance was used when relevant to compare means between different groups and better understand variations in the variables of interest.\\

This approach not only allowed for establishing the existence of correlations but also exploring the nature and significance of these relationships, providing a solid basis for subsequent interpretations and conclusions.

\subsection{Statistical Procedures}\label{procedimientos-estaduxedsticos}

\subsubsection{Application and Analysis of the Beck Depression Inventory}
After applying the scale to the sample, participants were categorized based on their scores: minimal or no depression, mild depression, moderate, and severe depression. This classification was the central focus for the following analyses.\\

\subsubsection{Normality Tests}
Before proceeding with comparative analyses, as part of the statistical processes, a crucial step was to analyze the normality of the sample. The \emph{Shapiro-Wilk} test implemented in \emph{`SciPy'} was used to check the data distribution. If the sample was normal, the following statistical tests could be performed.\\

\subsubsection{Variance Comparisons (ANOVA)}
An analysis of variance was used to compare dietary patterns among the different depression groups. If significant differences were identified, post-hoc tests were performed to determine where these differences resided.\\

\subsubsection{Correlation and Regression Analysis}
The relationship between the severity of depressive symptoms and dietary habits was explored using Pearson's correlation coefficient and regression analysis. This allowed for a better understanding of the nature and strength of these relationships.

\subsubsection{Data Visualization and Presentation}
With the support of the \emph{`Python'} programming language and \emph{`Jupyter Notebook'} for statistical analysis and Tableau for visualization, the data was presented in a way that highlighted key relationships and important findings, facilitating interpretation.

\subsubsection{Contextualized Interpretation of Results}
All results were interpreted considering reliability and statistical significance. The conclusions were discussed in the context of existing literature and practical implications for the general scientific context and specifically for the Guatemalan population.

\subsection{Inclusion of Key Aspects in the Analysis}\label{inclusiuxf3n-de-aspectos-clave-en-el-anuxe1lisis}

In this research, the integrity and accuracy of statistical analysis were fundamental. Therefore, several key aspects were incorporated to ensure the quality and reliability of the results and conclusions.\\

\subsubsection{Verification of Data Normality}
At each stage of the analysis, normality tests were conducted, such as the Shapiro-Wilk test to assess the distribution of the data where required. This verification was crucial to determine the suitability of selecting statistical tests given the normality of the sample and thus ensure the validity of the proposed interpretations.

\subsubsection{Evaluation of the Reliability of Measurement Tools}
It was essential and fundamental to validate the reliability of the tools used, such as the Beck Depression Inventory and the Dietary Questionnaire. This was done by calculating Cronbach's alpha coefficient, as high reliability ensured that the collected data was consistent and representative.

\subsubsection{Descriptive Statistics of the Sample}
Descriptive statistics of the sample, including means, medians, modes, ranges, and standard deviations, were provided. These statistics offered a clear and understandable view of the data to establish analyses based on a solid foundation for subsequent statistical tests.

\subsubsection{Contextualized Interpretation of Results}
All results were interpreted in light of contextualized hermeneutics based on respective reliability and statistical significance. Clear thresholds for statistical significance were also established, and the findings were discussed in relation to these criteria. This allowed for evaluating the strength and relevance of both the results and the presented conclusions. Additionally, the results were presented based on existing literature and relevant theories in the field. This provided relevant conclusions for the Latin American population, especially Guatemalans.


\section{Ethical Considerations}\label{consideraciones-uxe9ticas}

\subsection{Informed Consent}\label{consentimiento-informado}

Informed consent was obtained from all participants. To achieve this, they were provided with detailed information about the study's objectives, procedures, possible risks, and benefits. Additionally, it was ensured that they understood that their participation was entirely voluntary and that they had the right to withdraw from the study at any time without consequences.

\subsection{Confidentiality and Privacy}\label{confidencialidad-y-privacidad}

Confidentiality and privacy of all collected information were maintained. Participants' personal data were anonymized and kept confidential, ensuring their privacy and protection.

\subsection{Approval by an Ethics Committee}\label{aprobaciuxf3n-de-un-comituxe9-de-uxe9tica}

The research protocol was submitted for review and approval by an ethics committee, ensuring that it met ethical standards and current regulations.

\subsection{Impact and Benefit for Participants and the Community}\label{impacto-y-beneficio-para-los-participantes-y-la-comunidad}

The potential impact and benefits of the research for both the participants and the involved community were evaluated and described. Additionally, emphasis was placed on maximizing benefits and minimizing any potential risks.

\subsection{Budget}\label{presupuesto}

\begin{table}[H]
	\centering
	\begin{tabular}{>{\bfseries}l c}
		\toprule
		Item & Cost (Q) \\

		\midrule
		Band-Aids & 30 \\

		Needles & 150 \\

		Cotton & 20 \\

		Alcohol & 20 \\

		Gasoline & 200 \\

		Electricity and Internet & 100 \\

		\midrule
		Total & 520 \\

		\bottomrule
	\end{tabular}
	\caption{Budget for the study}
	\label{tab:presupuesto}
\end{table}

The budget will be covered by the principal investigators.

\subsection{Timeline}\label{cronograma}

\begin{table}[h!]
	\centering
	\begin{tabular}{@{}ll@{}}
		\toprule
		\textbf{Item}                & \textbf{Detail}     \\ \midrule
		Start Date              & February 2024         \\
		End Date        & June 2024           \\
		Duration                     & 5 months              \\ \bottomrule
	\end{tabular}
	\caption{Study Dates and Duration}
	\label{tab:fechas-duracion}
\end{table}

\section{Results}\label{resultados}

\subsection{Sociodemographic Data}

\begin{figure}[H]
	\centering
	\includegraphicsmax{freq.age.pdf}
	\caption{Age Frequency}
	\label{fig:Figure1}
\end{figure}

\begin{table}[H]
	\centering
	\resizebox{0.8\columnwidth}{!}{
		\begin{tabular}{@{}rrrr@{}}
			\toprule
			\textbf{Statistic}             & \textbf{Total Beck} & \textbf{Total Diet}  \\ \midrule
			\textbf{N}                       & 30.00                  & 30.00                   \\
			\textbf{Mean}                   & 19.00                  & 4.77                    \\
			\textbf{Median}                 & 17.00                  & 6.00                    \\
			\textbf{Standard Deviation}     & 9.28                   & 9.58                    \\
			\textbf{Minimum}                  & 7.00                   & -17.00                  \\
			\textbf{Maximum}                  & 42.00                  & 21.00                   \\
			\textbf{Skewness}               & 0.97                   & -0.22                   \\
			\textbf{Kurtosis}                & 0.14                   & -0.60                   \\
			\textbf{Shapiro-Wilk W}       & 0.90                   & 0.97                    \\
			\textbf{Shapiro-Wilk p-value} & 0.01                   & 0.63                    \\ \bottomrule
		\end{tabular}
	}

	\caption{Descriptive statistics of the variables}

	\label{tab:descriptives}
\end{table}

\subsubsection{Total Beck Scale}
The score on the Beck scale shows a non-normal distribution with a right skew. This suggests that there are more individuals with lower depression scores, but with a wide variability. The high standard deviation indicates significant differences between individual scores.

\subsubsection{Total Diet Questionnaire}
The distribution of diet scores is normal and has negative values, which are allowed by the questionnaire. The median is higher than the mean, indicating a slight tendency towards higher values.


\begin{figure}[H]
	\centering
	\begin{subfigure}[b]{\subfigwidth}
		\centering
		\includegraphics[width=\linewidth]{Box_Plot_of_dieta_total.pdf}
		\caption{Total Diet}
		\label{fig:BoxPlotDietaTotal}
	\end{subfigure}
	\hspace{0.5em}
	\begin{subfigure}[b]{\subfigwidth}
		\centering
		\includegraphics[width=\linewidth]{Box_Plot_of_beck_total.pdf}
		\caption{Beck Total}
		\label{fig:BoxPlotBeckTotal}
	\end{subfigure}

	\caption{Box and Whisker Plots of Variables}
	\label{fig:BoxPlots}
\end{figure}


% Please add the following required packages to your document preamble:
% \usepackage{booktabs}
% \usepackage{graphicx}
\begin{table}[]
	\resizebox{\columnwidth}{!}{%
		\begin{tabular}{@{}llrr@{}}
			\toprule
			\textbf{Variable \#1}           & \textbf{Variable \#2} & \multicolumn{1}{l}{\textbf{R$^2$}} & \multicolumn{1}{l}{\textbf{P\_Value}} \\ \midrule
			beck\_loss\_of\_pleasure        & diet\_fruit           & -0.489                                   & 0.006                                 \\
			beck\_suicidal\_thoughts        & diet\_fruit           & -0.480                                   & 0.007                                 \\
			beck\_pessimism                 & diet\_sugar           & 0.432                                    & 0.017                                 \\
			beck\_pessimism                 & age                   & -0.424                                   & 0.019                                 \\
			beck\_loss\_of\_interest        & diet\_soft\_drinks    & 0.418                                    & 0.021                                 \\
			diet\_red\_meat                 & blood\_pressure       & -0.415                                   & 0.022                                 \\
			beck\_difficulty\_concentrating & diet\_flour           & 0.414                                    & 0.023                                 \\
			beck\_pessimism                 & diet\_red\_meat       & 0.407                                    & 0.026                                 \\
			beck\_agitation                 & diet\_ginger          & -0.404                                   & 0.027                                 \\
			beck\_failure                   & diet\_olive\_oil      & 0.404                                    & 0.027                                 \\
			beck\_punishment                & diet\_fruit           & -0.398                                   & 0.030                                 \\
			beck\_crying                    & diet\_vegetables      & 0.396                                    & 0.030                                 \\
			beck\_dissatisfaction           & diet\_fruit           & -0.391                                   & 0.033                                 \\
			beck\_irritability              & diet\_red\_meat       & 0.388                                    & 0.034                                 \\
			beck\_irritability              & diet\_sugar           & 0.378                                    & 0.039                                 \\
			beck\_loss\_of\_interest        & diet\_flour           & 0.378                                    & 0.040                                 \\
			beck\_difficulty\_concentrating & diet\_nuts            & 0.376                                    & 0.040                                 \\
			beck\_irritability              & total\_diet           & -0.372                                   & 0.043                                 \\
			beck\_indecision                & diet\_soft\_drinks    & 0.372                                    & 0.043                                 \\
			diet\_flour                     & age                   & -0.367                                   & 0.046                                 \\
			beck\_agitation                 & diet\_alcohol         & 0.366                                    & 0.047                                 \\
			beck\_loss\_of\_interest        & diet\_fruit           & -0.365                                   & 0.047                                 \\
			beck\_self\_disesteem           & age                   & -0.360                                   & 0.050                                 \\
			beck\_appetite\_changes         & diet\_soft\_drinks    & 0.354                                    & 0.055                                 \\
			beck\_punishment                & total\_diet           & -0.351                                   & 0.057                                 \\ \bottomrule
		\end{tabular}%
	}
	\caption{Correlation Table}
	\label{tab:tableOfCorr}
\end{table}




The table \ref{tab:tableOfCorr} shows the correlations between various consumption variables and different emotional or psychological symptoms, along with the associated p-values indicating the statistical significance of these correlations.

There is a significant negative correlation between fruit consumption and loss of pleasure, suggesting that higher fruit consumption is associated with a lower loss of pleasure. There is also a significant negative correlation between fruit consumption and suicidal thoughts, indicating that higher fruit consumption is associated with a lower frequency of suicidal thoughts.

Other observed correlations include a positive correlation between sugar consumption and pessimism, suggesting that higher sugar consumption is associated with higher levels of pessimism. A positive correlation between soda consumption and loss of interest indicates that higher soda consumption is associated with greater loss of interest.

Additionally, there is a positive correlation between flour consumption and difficulty concentrating, suggesting that higher flour consumption is associated with greater difficulty concentrating. There is also a positive correlation between red meat consumption and pessimism, indicating that higher red meat consumption is associated with higher levels of pessimism.

There is a significant negative correlation between fruit consumption and feelings of punishment, indicating that higher fruit consumption is associated with lower feelings of punishment. There is also a negative correlation between fruit consumption and dissatisfaction, suggesting that higher fruit consumption is associated with lower levels of dissatisfaction.

A positive correlation was observed between red meat consumption and irritability, indicating that higher red meat consumption is associated with greater irritability.

Most of the negative correlations involve fruit consumption, which could suggest a protective effect of fruit consumption against certain negative psychological symptoms. The positive correlations between sugar and soda consumption with negative symptoms such as pessimism and loss of interest could indicate the need to moderate the consumption of these products to improve emotional well-being.

All mentioned correlations are significant with a p-value \textless{} 0.05, which supports the robustness of these associations. These insights can be used to recommend dietary adjustments as part of a holistic approach to improving emotional and mental well-being.

Cronbach's $\alpha$ is a measure of the internal consistency of a scale. Values between 0.7 and 0.8 are generally considered acceptable, suggesting that the scale has adequate reliability. In this case, a value of 0.741 indicates that the items in the diet questionnaire are reasonably correlated and measure the same underlying construct.

McDonald's $\omega$ is another measure of internal consistency and is often considered a more accurate estimate than Cronbach's $\alpha$. A value above 0.9 is considered excellent, indicating high reliability of the scale. In this case, a value of 0.904 suggests that the questionnaire is very reliable and that the items are consistent in measuring the construct.

Both values indicate good internal consistency, although McDonald's $\omega$ is notably higher than Cronbach's $\alpha$. This can occur when the items have different factor loadings, and McDonald's $\omega$, which takes these differences into account, provides a more accurate estimate of reliability. The diet questionnaire we developed is reliable for use in research and practical applications, as both reliability metrics exceed the generally accepted thresholds.


\subsection{Beck Depressive Symptoms}
\begin{figure}[H]
	\centering
	\includegraphicsmax{sintomasDepresivosBeckGraph.pdf}
	\caption{Heat Map: Depressive Symptoms}
	\label{fig:Figure2}
\end{figure}
\vspace{-1em} % Adjust vertical space

\subsection{Dietary Habits of Young Guatemalan Women}
\begin{figure}[H]
	\centering
	\includegraphicsmax{dietGraph.pdf}
	\caption{Heat Map: Dietary Questionnaire}
	\label{fig:Figure3}
\end{figure}

\section{Discussion of Results}\label{discusiuxf3n-de-resultados}

This study aimed to analyze the relationship between dietary habits and the severity of depressive symptoms in young Guatemalan women. The results provide significant evidence of these interactions and offer valuable insights into understanding the relationship between diet and depression in this specific population.

\subsection{Summary of Main Findings}\label{resumen-de-hallazgos-principales}

Our study revealed significant correlations between certain dietary patterns and specific depressive symptoms. In particular, a strong negative association was found between fruit consumption and various depressive symptoms, including loss of pleasure $(r = -0.49, p = 0.006)$ and suicidal thoughts $(r = -0.48, p = 0.007)$. On the other hand, positive correlations were observed between the consumption of processed foods (such as sugars, sodas, and flours) and symptoms such as pessimism, loss of interest, and difficulty concentrating.\\

\subsection{Interpretation of Results}\label{interpretaciuxf3n-de-resultados}

\subsubsection{Protective Effect of Fruit Consumption}

The negative correlation between fruit consumption and depressive symptoms suggests a possible protective effect of a fruit-rich diet against depression. This could be attributed to the nutrients and antioxidants present in fruits, which may have a positive impact on mental health. The particularly strong association with the reduction of suicidal thoughts is a notable finding that warrants further investigation.\\

This relationship could be explained by several mechanisms:

\begin{enumerate}
	\item Antioxidants present in fruits may reduce oxidative stress, which has been associated with depression.\\

	\item Fruits are rich in essential vitamins and minerals for optimal neurological functioning.\\

	\item The dietary fiber in fruits may positively influence the gut microbiota, which in turn affects the gut-brain axis.
\end{enumerate}

\subsubsection{Negative Impact of Processed Foods}\label{impacto-negativo-de-alimentos-procesados}

The positive correlations between the consumption of processed foods and depressive symptoms support the hypothesis that a diet high in refined sugars and saturated fats may contribute to the development or exacerbation of depressive symptoms. Possible mechanisms for this relationship include:

\begin{enumerate}
	\item Processed foods can cause blood glucose spikes, which can affect mood.
	\item These foods often lack essential nutrients for mental health.
	\item Excessive consumption of processed foods can lead to obesity, which has been associated with a higher risk of depression.
\end{enumerate}



\subsection{Comparison with Existing Literature}\label{comparaciuxf3n-con-literatura-existente}

Our findings on the protective effect of fruits are consistent with previous studies that have found associations between fruit and vegetable consumption and a reduced risk of depression. For example, a meta-analysis conducted by \parencite{liuFruitVegetableConsumption2016} found that higher fruit and vegetable consumption was associated with a lower risk of depression. However, our study provides specific evidence of the relationship between fruit consumption and specific depressive symptoms in a little-studied population: young Guatemalan women.\\

The positive association between processed foods and depressive symptoms also aligns with previous research linking Western diets (high in processed foods) with a higher risk of depression. For example, \parencite{laneUltraProcessedFoodConsumption2022} found that a diet characterized by processed foods was associated with a higher likelihood of depression and anxiety in women. Nonetheless, our study provides a more detailed view by examining correlations with specific symptoms. Additionally, several other studies have been published evaluating the relationship between ultra-processed food consumption and depression, as well as other mental disorders. Our study included a total of 17 observational studies (n = 385,541); 15 cross-sectional and 2 prospective. Higher consumption of ultra-processed foods was cross-sectionally associated with greater odds of depressive and anxiety symptoms, both when these outcomes were evaluated together (odds ratio of common mental disorder symptoms: 1.53, 95\% CI 1.43 to 1.63) and separately (odds ratio of depressive symptoms: 1.44, 95\% CI 1.14 to 1.82; and, odds ratio of anxiety symptoms: 1.48, 95\% CI 1.37 to 1.59). Additionally, a meta-analysis of prospective studies showed that greater intake of ultra-processed foods was associated with an increased risk of subsequent depression (hazard ratio: 1.22, 95\% CI 1.16 to 1.28). Although we found evidence of associations between ultra-processed food consumption and adverse mental health, rigorously designed prospective and experimental studies are needed to better understand the causal pathways.

\subsection{Implications}\label{implicaciones}

These results have important implications for both clinical practice and public health:

\begin{enumerate}
	\item Dietary interventions: The findings suggest that dietary interventions, particularly increasing fruit consumption and reducing processed foods, could be effective strategies for the prevention and management of depression in young women.
	\item Public health policies: These results could inform public health policies aimed at improving mental health through the promotion of healthy diets. For example, nutritional education programs focused on increasing fruit consumption and reducing the intake of processed foods could be implemented.
	\item Integrated approach: The association between diet and depressive symptoms reinforces the importance of an integrated approach to depression treatment that considers both psychological and nutritional factors. Mental health professionals might consider including dietary recommendations as part of their treatment plans.
	\item Prevention: Since our study focused on young women, the results suggest that early dietary interventions could play an important role in preventing depression in this population.
\end{enumerate}

\section{Limitations}\label{limitaciones}

It is important to acknowledge the limitations of this study:

\begin{enumerate}
	\item Sample size: With 30 participants, the sample size is relatively small, which may limit the generalizability of the results. Future studies should consider larger samples to increase statistical power.
	\item Cross-sectional design: The cross-sectional design of the study does not allow for establishing causal relationships between diet and depressive symptoms. Longitudinal studies are needed to determine the direction of causality.
	\item Specific population: The study focused on young Guatemalan women, which may limit the applicability of the results to other populations. Studies in different demographic groups are needed to confirm whether these findings are generalizable.
	\item Definition of dietary patterns: As observed in the literature, the lack of a standardized definition of "healthy diet" can make comparison between studies difficult. In our case, we focused on specific foods rather than general dietary patterns, which may limit comparability with other studies.
	\item Variability in depression measurement: Although we used the Beck scale, which is widely validated, the literature notes that variability in depression measures across studies can make it difficult to compare results.
	\item Confounding factors: Although several demographic and lifestyle factors were controlled for, there may be other unmeasured confounding factors that could influence the relationship between diet and depression.
\end{enumerate}


\section{Recommendations for Future Research}\label{recomendaciones-para-futuras-investigaciones}

Based on our findings and limitations, we recommend:
\begin{enumerate}
	\item Conduct longitudinal studies to establish causal relationships between dietary patterns and depressive symptoms.
	\item Investigate the biological mechanisms underlying the relationship between fruit consumption and the reduction of depressive symptoms.
	\item Explore the effectiveness of specific dietary interventions in the prevention and treatment of depression.
	\item Expand the study to more diverse and larger populations.
	\item Standardize the definition and measurement of dietary patterns to facilitate comparison between studies.
	\item Use multiple measures of depression, including structured clinical interviews, to obtain a more comprehensive assessment of depressive symptoms.
	\item Investigate the interaction between diet and other lifestyle factors, such as exercise and sleep, in relation to depression.
\end{enumerate}

\section{Conclusion}\label{conclusiuxf3n}

This study provides important evidence on the relationship between dietary patterns and depressive symptoms in young Guatemalan women. The findings underscore the potential protective role of a fruit-rich diet and the possible negative effects of processed foods on mental health. Although more research is needed, these results suggest that dietary interventions could be a valuable component in prevention and treatment strategies for depression.\\

The complexity of the relationship between diet and depression evidenced in this study underscores the need for a multidisciplinary approach to depression research and treatment. As we advance in understanding these interactions, it is crucial that mental health professionals, nutritionists, and public health policymakers work together to develop comprehensive strategies that address both the nutritional and psychological aspects of mental health.

%\nocite{*}
\printbibliography

\end{document}


%
%%%
%%% Copyright (C) 2019 by Daniel A. Weiss <daniel.weiss.led at gmail.com>
%%%
%%% This work may be distributed and/or modified under the
%%% conditions of the LaTeX Project Public License (LPPL), either
%%% version 1.3c of this license or (at your option) any later
%%% version.  The latest version of this license is in the file:
%%%
%%% http://www.latex-project.org/lppl.txt
%%%
%%% Users may freely modify these files without permission, as long as the
%%% copyright line and this statement are maintained intact.
%%%
%%% This work is not endorsed by, affiliated with, or probably even known
%%% by, the American Psychological Association.
%%%
%%% This work is "maintained" (as per LPPL maintenance status) by
%%% Daniel A. Weiss.
%%%
%%% This work consists of the file  apa7.dtx
%%% and the derived files           apa7.ins,
%%%                                 apa7.cls,
%%%                                 apa7.pdf,
%%%                                 README,
%%%                                 APA7american.txt,
%%%                                 APA7british.txt,
%%%                                 APA7dutch.txt,
%%%                                 APA7english.txt,
%%%                                 APA7german.txt,
%%%                                 APA7ngerman.txt,
%%%                                 APA7greek.txt,
%%%                                 APA7czech.txt,
%%%                                 APA7turkish.txt,
%%%                                 APA7endfloat.cfg,
%%%                                 Figure1.pdf,
%%%                                 shortsample.tex,
%%%                                 longsample.tex, and
%%%                                 bibliography.bib.
%%%
%%%
%%% End of file `./samples/longsample.tex'.